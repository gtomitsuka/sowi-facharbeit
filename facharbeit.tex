\documentclass[a4paper,draft]{report}
\usepackage[ngerman]{babel}

\begin{document}
\title{Wurden die richtigen Lehren aus der Finanzkrise 2008 gezogen?}
\subtitle{Eine Analyse mit Hinblick auf den Finanzsektor}
\author{Gabriel Tomitsuka}
\date{Oktober 2019}
\maketitle

\tableofcontents

\section{Einleitung}
Am 2. April 2007 hat der Real Estate Investment Trust (REIT) namens New Century Insolvenz erklärt. Wenige Monate davor hatte er eine Marktkapitalisation von $1.75 Milliarden. Früher wäre kaum vorstellbar gewesen, dass ein REIT dieser Größe einen gewaltigen Einfluss auf die globalen Kapitalmärkte haben kann - allerdings wurden in den 90ern hypothekenbesicherte Wertpapiere (besser bekannt als Mortgage-Backed Security oder MBS) entwickelt - diese haben alles verändert. Weniger als zwei Jahre später erklärte die zuvor als "Too Big To Fail" verstandene Investmentbank Lehman Brothers Insolvenz, und Banken wie Dresdner Bank und Merill Lynch mussten aufgekauft werden, um das selbe Ende zu vermeiden. Die Weltwirtschaftskrise die dadurch entstanden ist war die größte seit der Großen Depression von 1929 bis 1933-34 und verständlicherweise versuchen seitdem Regulatoren, eine ähnliche Krise mit allen Mitteln zu vermeiden.

In dieser Facharbeit werde ich untersuchen, inwiefern ihnen das gelingt, und ob möglicherweise ungewollte Konsequenzen

\end{document}
http://web.mit.edu/rsi/www/pdfs/new-latex.pdf
http://web.mit.edu/rsi/www/pdfs/bibtex-format.pdf
http://www.ir.rochelleterman.com/sites/default/files/Helleiner%202011.pdf
