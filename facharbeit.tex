\documentclass[a4paper,draft]{report}
\usepackage[ngerman]{babel}

\begin{document}
\title{Wurden die richtigen Lehren aus der Finanzkrise 2007-2009 gezogen?}
\author{Gabriel Tomitsuka}
\date{Oktober 2019}
\maketitle

\tableofcontents

\section{Einleitung}
Am 2. April 2007 hat der Real Estate Investment Trust (REIT) namens New Century Insolvenz erklärt.
Wenige Monate davor hatte er eine Marktkapitalisation von \$1.75 Milliarden.
<story zu anfang>

\subsection{Themawahl}
Ich habe mich oft gefragt, wie ein Markt, der von kleinen, amerikanischen Regionalbanken mit staatlicher Unterstüzung
dominiert wird zum Fall von als "Too Big To Fail" verstandene Investmentbanken wie Lehman Brothers und Merill Lynch
führen konnte. Besonders interessant fand ich dass auch Banken in anderen Kontinenten (wie bspw. Dresdner Bank) dadurch 
aufgekauft werden mussten. Ich wollte genauer ermitteln, wie stark diese Vernetzung damals war und heute ist, und ob
die kurzfristige Maßnahmen durch Zentralbanker, Banken/Fonds und Regulatoren womöglich eine noch schlimmere Krise
vermieden haben.

Allerdings werde ich mich nicht besonders lange mit der geschichtlichen Fragestellung beschäftigen, ob die kurzfristige
Maßnahmen wie geplant wirken, sondern viel mehr mit der Fragestellung, ob die mittel- und langfristigen Maßnahmen die Welt
auf die naechste Rezession richtig vorbereitet hat - Alle Haendler und Investmentbanker mit denen ich gesprochen habe waren
der Ansicht, dass wir uns spaet im Marktzyklus befinden, und die Fruehindikatoren zeigen aehnliche Signale.
Daher ist jetzt besonders wichtig, zu ueberpruefen, inwiefern wir bereit sind.

\subsection{Vorgang}
Als erstes werde ich die Instrumente untersuchen, die für die Krise verantwortlich gemacht werden - Asset-Backed Securities (ABS), 
Collateralized Debt Obligations (CDOs) und Credit Default Swaps (CDS). Daraufhin werde ich <verb>, und die Frage erörtern, ob es andere 
Instrumente oder Asset Klassen gibt, die eine besondere Wirkung auf die Krise hatten, die in der Oeffentlichkeit nicht angekommen ist.

\end{document}
http://web.mit.edu/rsi/www/pdfs/new-latex.pdf
http://web.mit.edu/rsi/www/pdfs/bibtex-format.pdf
http://www.ir.rochelleterman.com/sites/default/files/Helleiner%202011.pdf
Quelle 1: https://www.federalreserve.gov/econresdata/releases/mortoutstand/mortoutstand20090331.htm
Quelle 2: https://web.archive.org/web/20090419051155/http://www.sifma.org/uploadedFiles/Research/Statistics/SIFMA_USBondMarketOutstanding.pdf

ä
ö
ü
ß