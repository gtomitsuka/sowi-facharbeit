\documentclass[a4paper,draft,11pt]{report}
\usepackage[T1]{fontenc}
\usepackage[english,ngerman]{babel}
\usepackage[utf8]{inputenc}
\usepackage{titlesec}
\usepackage[onehalfspacing]{setspace}
\usepackage[autostyle=true]{csquotes}
\usepackage{geometry}
\usepackage{fancyhdr}
\usepackage[
  backend=bibtex,
  style=authoryear,
  firstinits=true,
  uniquename=init,
  sortlocale=de_DE
]
{biblatex}
\DeclareNameAlias{sortname}{last-first}
\DeclareFieldFormat*{title}{#1}
\DefineBibliographyStrings{ngerman}{
   andothers = {{et\,al\adddot}},            
}
\addbibresource{ref.bib}
\geometry{
  left=25mm,
  right=35mm,
  top=30mm,
  bottom=20mm,
  bindingoffset=5mm
}
\titleformat{\chapter}[display]
  {\normalfont\bfseries}{}{0pt}{\Large}
\titlespacing{\chapter}{0pt}{25pt}{10pt}

\renewcommand{\rmdefault}{phv} % Arial
\renewcommand{\sfdefault}{phv} % Arial
\renewcommand{\headrulewidth}{0.1pt} 
\interfootnotelinepenalty=10000

\usepackage{etoolbox}
\makeatletter
\patchcmd{\chapter}{\if@openright\cleardoublepage\else\clearpage\fi}{}{}{}
\makeatother

\begin{document}
\title{Wurden die richtigen Lehren aus der
    Finanzkrise 2007-2009 gezogen?}
\author{%
    Facharbeit im Leistungskurs Sozialwissenschaften \\
    Ravensberger-Gymnasium Herford \\ \\
    \textit{eingereicht bei} \\
    Herrn Visser \\ \\
    \textit{vorgelegt von} \\
    Gabriel Tomitsuka 
    }
\date{Herford, Oktober 2019}
\maketitle
\tableofcontents
\newpage
\chapter{Einleitung}
\section{Themawahl}
Laut dem Gesch\"aftsf\"uhrer der Investmentbank J.P. Morgan
Chase Jamie Dimon war die Woche des 15. Oktober 2008, also ab dem
Tag, wo Lehman Brothers Insolvenz angemeldet hat,
mit der Woche des \enquote{Great Crashes} ab dem 24. Oktobers 1929
von einer finanzwirtschaftlichen Perspektive vergleichbar,
allerdings hat die rechtzeitige Reaktion der Regierungen und
Regulatoren das Schlimmste vermieden \parencite{dimonyt}.

Ich habe mich oft gefragt, wie ein Markt, der
von kleinen, amerikanischen Regionalbanken mit
staatlicher Unterstützung
dominiert wird, zum Fall von als \enquote{Too Big To
Fail} verstandene Investmentbanken wie Lehman
Brothers und Merill Lynch
führen konnte. Besonders interessant fand ich,
dass auch Banken in anderen Kontinenten (wie
bspw. die Dresdner Bank) dadurch
aufgekauft werden mussten.

Ich wollte genauer
ermitteln, wie stark diese Vernetzung damals
war und heute ist, und ob
die kurzfristigen Maßnahmen durch
Zentralbanker, Banken/Fonds und Regulatoren
womöglich eine noch schlimmere Krise
vermieden haben -- also ob die Behauptung Dimons
stimmt.

Allerdings werde ich mich nicht besonders
lange mit einer historischen Betrachtung
beschäftigen, ob die kurzfristigen
Maßnahmen wie geplant wirkten, sondern viel
mehr mit der Fragestellung, ob die mittel- und
langfristigen Maßnahmen die Welt
auf die nächste Rezession richtig vorbereitet
haben --- alle Händler und Investmentbanker, mit
denen ich mich im Rahmen meiner Facharbeit ausgetauscht habe, waren
der Ansicht, dass wir uns spät im Marktzyklus
befinden, und die Frühindikatoren zeigen
ähnliche Signale.

Daher ist es besonders wichtig zu
überprüfen, inwiefern das gesamtwirtschaftliche System auf die nächste Rezession vorbereitet ist.
\section{Vorgang}
Die Ursachen für die Finanzkrise 2008 können 
in zwei Gruppen unterteilt werden: Ursachen der Immobilienkrise
in den Vereinigten Staaten und die finanzwirtschaftliche
Ursachen der globalen Finanzkrise.

\"Uber die realwirtschaftlichen Ursachen besteht weithin
Einigkeit innerhalb der Wirtschaftswissenschaften, daher
werde ich verk\"urzt auf sie eingehen.

Um die finanzwirtschaftliche Ursachen zu erläutern, werde
ich erst auf die theoretische Grundlage des Finanzsystems
nach den Deregulierungen in den 1970er Jahren eingehen,
sowie die der Bankentheorie. Daraufhin werde ich die 
Instrumente untersuchen, die dazu beigetragen haben,
dass diese theoretische Grundlage nicht mehr galt --- 
dazu geh\"oren Asset-Backed Securities (ABS),
Collateralized Debt Obligations (CDO) und
Credit Default Swaps (CDS). Daraufhin werde
ich untersuchen, wie genau sie für die
Krise mitverantwortlich sind. Ferner werde ich
die Frage erörtern, ob es andere Instrumente
oder Anlageklassen gibt, die eine
besondere Wirkung auf die Krise hatten, die in
der Öffentlichkeit nicht angekommen ist.

Daraufhin werde ich die kurzfristige Reaktion
der Regulatoren und Zentralbanker ab 2008
untersuchen. Genauer werde ich auf
Sofortmaßnahmen wie die kontroverse
Bankenrettung, (bis heute anhaltende)
Zinssenkungen, <etc> eingehen.
Schließlich werde ich auf die mittel- und
langfristigen Maßnahmen, inkl. Basel
III und CRD eingehen.

Schließlich werde ich mit Bezug auf
privaten Gesprächen mit Händlern,
Risikoverwaltern und Investmentbankern 
sowie \"offentliche Interviews mit F\"uhrungskr\"aften
aus dem Finanzsektor bzw.
Ver\"offentlichungen finanzieller Institutionen
erörtern, ob die Maßnahmen die europäische
Wirtschaft ausreichend auf die
Herausforderungen der Zukunft vorbereiten.

\chapter{Amerikanische Immobilienkrise}
Über die realwirtschaftliche Ursachen der Krise besteht
weithin Einigkeit in den Wirtschaftswissenschaften: 
der Kreditboom in den USA und die Immobilienblase.
Im Folgenden werde ich erläutern, wie diese entstanden
sind und wie sie zur Krise beigetragen haben.

\section{Der Hypothekenmarkt}
Aus den f\"unfundsiebzig Millionen H\"ausern im
Privateigentum 2007 in den USA, wurden auf etwa f\"unfzig Millionen
Hypotheken aufgenommen (Buffett, 2018).
In der Zeit zwischen 2002 und 2007 hat sich das
Schulden zu Nationaleinkommen-Verhältnis
der USA von 3,75:1 auf 4,75:1 erhöht.
Um die letzte Erhöhung der Schulden in
dieser Größenordnung zu erreichen, hat man davor
das gesamte Jahrzehnt der 1990er
gebraucht \parencite[S. 195f.]{acharyar}.

Durch das Schaubild in Appendix 1 wird deutlich,
dass der Wachstum der Immobilienpreise keine Verbindung
mit den finanziellen Zugewinnen der Amerikaner hatte,
und zumeist spekulativ war. 

Laut dem US-amerikanischen
Großinvestor Warren Buffett war die Einstellung
des durchschnittlichen amerikanischen
B\"urgers zu der Zeit,
dass die Immobilienpreise ewig weitersteigen w\"urden,
und basierend auf dieser Erwartung durch das Haus
gesicherte Kredite aufgenommen haben (Buffett, 2018).

\subsection{Der Subprime-Markt}
Viele Immobilenkredite wurden von Menschen, die selbst
in einer Niedrigzinsumgebung sich die Immobilie nicht leisten
konnten, die sie erworben haben.
Dazu kommen die strategischen Anreize, um Kreditnehmer
mit wenig Bonit\"at (Subprime) zu erwerben. Dazu z\"ahlen
Kredite ohne Einkommensbeweise (siehe Appendix II), 
und hybride Kredite mit Teaserraten. Sie wurden in "2/28\"\-- oder
"3/27\"\--Modellen verkauft, die sehr attraktive, fixe Zinsraten
f\"ur die ersten zwei bzw. drei Jahre hatten, und
f\"ur die letzten achtundzwanzig bzw.
siebenundzwanzig exorbitant hohe Zinss\"atze\footnote{
  wie solche Kredite m\"oglich waren, wird in 3.3 erl\"aurtert}.
Diese waren so konstruiert, dass sie in ausgesprochen vielen
F\"allen den Kreditnehmer zur Insolvenz f\"uhrten --
nach diesen wenigen Jahren hatte ein Subprime--Kreditnehmer
keine Wahl. Entweder stiegen die Immobilienpreise deutlich
und er kann die Immobilie neu finanzieren, oder er muss
Insolvenz erkl\"aren \parencite[S. 208]{acharyar}.

Unter diesen Umst\"anden wird deutlich, dass diese eine
Umgebung war, wo selbst eine verringterte Wachstumsrate
der Immobilienpreise ausgereicht h\"atte,
um den Binnenmarkt in den USA zu schaden -- wenn sich
Menschen darauf verlassen, dass Immobilienpreise
steigen m\"ussen, damit sie ihr Haus \"uberhaupt behalten
d\"urfen, ist ein Sturz mit dramatischen Folgen verbunden.

\section{R\"uckgang ab 2006}
Allerdings passierte ab November 2006 genau das.
Das Wachstum des realen HPI (House Price Index,
Purchase Only) wurde Anfang 2006 langsamer, und
ab November 2006 sank es (Appendix 3).

In August 2006 ist der Anteil der insolventen
Immobilien, die an Subprime Kreditnehmer verkauft wurden,
auf 7.74\% gestiegen (vgl. August 2005: 5,53\%).

Kurz daraufhin waren die ersten spezialisierte
Kreditgeber im Bereich Subprime ebenfalls insolvent - 
Ownit Mortgage Solutions Inc. hat Dezember 2006 Chapter 11
Bankruptcy Protection\footnote{Demnach kann eine insolvente
Firma eine Sanierung bzw. Reorganisation vorschlagen,
um ihre Kredite versp\"atet zahlen zu k\"onnen, und somit
das Gesch\"aft zu retten. In Ownits Fall war dieser Plan nicht
erfolgreich.} beantragt. Ownit war der elftgrößte Kreditgeber
im Subprime-Sektor \parencite{wsjdoss}. 

Am 12. M\"arz 2007 hat der zweitgr\"oßte Kreditgeber im Bereich
Subprime, New Century Financial Corporation, aufgeh\"ort, neue
Kredite zu geben. Sie meldeten Insolvenz einen Monat sp\"ater
an\footnote{Die Insolvenz hatte unterschiedliche Gr\"unde -
zus\"atzlich hat New Century keine Investoren mehr gefunden
und sie haben mehrere Sammelklagen verloren.}. 2007 haben sie noch
etwa 60 Milliarden Dollar an Krediten vergeben \parencite{nytcres}.

Subprime-Kreditnehmer, die hybride Kredite in "2/28\"\-- oder
"3/27\"\--Modellen gekauft haben, wollten zu dieser Zeit ihre Kredite neu finanzieren.
Allerdings war dies nicht m\"oglich, da eine Refinanzierung nach
Wertverlust der Immobilie keinen Sinn ergibt. Dies  f\"uhrte zu
einer Welle von Insolvenzen, die fast zwangsl\"aufig sonstige
Wirkungen auf den US-Binnenmarkt haben w\"urde.
Die durchschnittliche amerikanische Familie, dessen Haus
hypothekarisch belastet war und dessen Haus etwa 35\% ihres
Gesamtverm\"ogens betrug, hat 2008 aus diesen Gr\"unden deutlich weniger
konsumiert als in den Vorjahren -- eine Rezession war zu erwarten. 
\parencite[196]{acharyar}

\chapter{Globale Finanzkrise}
Warum diese Faktoren aber zu einer Finanzkrise gef\"uhrt haben,
durch die viele der gr\"oßten Finanzinstitutionen
der Welt gefallen sind und Kapitalm\"arkte eingefroren haben,
ist allerdings viel weniger klar. Um die am weitesten verbreitete
Theorie zu erkl\"aren, ist notwendig, zu verstehen, warum das
System versagt hat. Daf\"ur werde ich erst die alte Finanzordnung vor 
den 1980ern erkl\"aren, um daraufhin 

\section{Theoretische Grundlage des modernen Bankings}


\section{Die alte Finanzarchitektur}
Nach der Weltwirtschaftskrise 1929 war die mehrheitliche politische
Meinung, dass unregulierte
Finanzm\"arkte intrinisisch instabil sein und stark reguliert
werden m\"ussen, um schwere wirtschaftliche Krisen und politische bzw. gesellschaftliche
Unruhe zu vermeiden. \parencite[S. 563f.]{crottycam}.

Das Bretton-Woods-System,
entwickelt nach den Vorstellungen von John Maynard Keynes,
Hyman Minsky und Harry Dexter White, hat eine neue globale W\"ahrungs-
und Handelsordnung in Kraft gesetzt, welche Regierungseingriffe
in W\"ahrungs- und Finanzkrisen erleichtert, und sollten es
vereinfachen, dass Regierungen Maßnamen (wie z.B. gegen die Arbeitslosigkeit) ergreifen
k\"onnen, ohne sich Sorgen um ihre Zahlungsf\"ahigkeit machen zu m\"ussen.

Durch die IMF und die Weltbank sollten Liquidit\"atsprobleme
vermieden werden und Regierungen und Regulatoren langfristig handlungsf\"ahiger werden
\parencite[S. 31f.]{bordo}.

\section{Die neue Finanzarchitektur}
Wirtschaftliche Turbulenzen in den 1970er Jahren f\"uhrten
zu einen Paradigmenwechsel: die \"Uberzeugung, dass staatliche
Vorschriften im Finanzsektor mehr Schaden als Nutzen anrichten.
\"Uber die Zeit ersetzte die neoklassische Theorie den 
Keynesianismus und die enge regulatorische Umgebung wurde gelockert
-- Geschäftsbanken sollten wenig reguliert werden, Investmentbanken
noch weniger und Schattenbanken (bspw. Hedge- und Private-Equity-Fonds) kaum.

Die Maßnahmen der Deregulierung in den USA fingen in 1980 an. Der
\textit{Depository Institutions Deregulation and Monetary Control Act (DIDMCA)}
entfernte die Zinsobergrenze und erteilte Banken und anderen
finanziellen Institutionen die Erlaubnis, Darlehen mit variablen
 Zinsen zu vergeben\footnote{Wegen DIDMCA k\"onnen bis heute
 Institutionen, die Kurzzeitkredite an Subprime-Kreditnehmer
 vergeben, Zinss\"atze bis 700\% im Jahr zu verlangen.}
\parencite[6--8]{sherman2009short}.

Zwar hat DIDMCA die Dynamik des Kreditmarktes in Amerika nachhaltig
ver\"andert, allerdings war die einflussreichste Deregulierungsmaßname
die Lockerung und anschließende Aufhebung des
\textit{Glass-Steagall Act of 1933}.
Die gesetzliche Trennung zwischen Geschäfts- und Investment-Banken
beabsichtigte eine Vermeidung von Interessenkonflikten und 
\"uberm\"aßiger Risikobereitschaft durch Geschäftsbanken.
In der Praxis bedeutete das, dass eine Gesch\"aftsbank keine 
Verbriefungen b\"urgen oder verkaufen konnte und ausschließlich
erstklassige Wertpapiere f\"ur sich kaufen konnte.

In 1986 hat die \textit{Federal Reserve} Glass-Steagall neu interpretiert:
Geschäftsbanken durften bis zu 5\% ihres Umsatzen durch Investmentbanking
erzeugen.

In 1996 ist die \textit{Federal Reserve} einen Schritt weiter gegangen:
Bankholdinggesellschaften durften bis zu 25\% ihres Umsatzes durch
Investmentbanking generieren, was Glass-Steagall in der Praxis aufhebt,
da nahezu alle Institutionen unter der 25\%-Grenze bleiben k\"onnten.
In 1999 hat die Clinton-Regierung Glass-Steagall offiziell aufgehoben.

Ein weiterer Schritt war die \textit{
  Riegle-Neal  Interstate  Banking  and  Branching Efficiency Act}, die
die Restriktionen des Bankings \"uber die Grenzen eines US-Bundesstaats
entfernt hat. Durch Fusionen sank die Zahl der amerikanischen 
Banken um 27\% zwischen 1990 und 1998 \parencite[8--12]{sherman2009short}.

\subsection{Strukturelle Probleme}
Der postkeynesische \"Okonom James Crotty behauptet, die neue Finanzstruktur
habe auch in der Theorie Probleme. Ein Hauptargument der Neoklassik sei, dass
Kapitalm\"arkte Wertpapiere richtig bepreisen und das Risiko gut einsch\"atzen
k\"onnen. Marktteilnehmer seien dementsprechend in der Lage,
richtige Entscheidungen entsprechend ihres Risikoprofils zu treffen.
Somit seien Finanzkrisen sehr unwahrscheinlich, da Marktteilnehmer nur so viel
Risiko aufnehmen w\"urden, wie sie tragen k\"onnen. Allerdings soll dieses Argument
keine empirische Beweise haben und auf unrealistische Annahmen basiert sein 
\parencite[563--565]{crottycam}.

\section{Finanzielle Innovation und strukturierte Produkte}
Finanzielle Innovation in Verbriefungen hat zu Instrumenten gef\"uhrt,
die so komplex sind, dass sie zwangsl\"aufig intransparent sind, und dementsprechend
kann die Theorie des fairen Preises durch den Markt auf komplexe strukturierte Produkte
nicht angewandt werden.

Eine MBS besteht aus mehreren Tausenden Hypotheken
und eine hypothekenbesicherte CDO (Collateralized Debt 
Obligation) aus bis zu 150 MBS. Eine CDO wird in viele
Tranchen nach Risiko geteilt. Die 

Eine \textit{CDO-Squared}
ist eine CDO, die Tranchen vieler CDOs als Sicherungsgegenst\"ande verwendet.
Eine \textit{CDO-Squared} ist nahezu unm\"oglich, fair zu bepreisen. Das liegt teilweise
daran, dass eine gleiche MBS in mehreren CDOs vielfach vorkommen kann.

Laut Prof. Dr. George Chacko der Harvard Business School gibt es kein
allgemein verwendbares Modell bzw. Formel die diese Rechnungen durchf\"uhren kann
und praktisches Nutzen hat \parencite[226]{chackocred}. Ratingagenturen lassen
hochkomplexe Simulationen tagelang laufen, um das Risiko eines \textit{CDO-Squared}
einsch\"atzen zu k\"onnen.

Produkte wie CDOs konnten w\"ahrend des Booms zwar verkauft werden, da  
es ausreichend Spekulatoren gab und die Marktbedingungen historisch freundlich waren.

Allerdings hat man, als die Krise angefangen hat, gemerkt, dass eigentlich niemand weiß,
wie viel CDOs tats\"achlich wert sind.

\subsection{Einfluss auf Europa}
EU sauce
\section{\foreignquote*{english}{Skin in the Game} - Versagen des Risikomanagements}
\section{VAR-Modellierung}
\chapter{Kurzfristige Maßnahmen der Regulatoren}

\chapter{Mittel- und langfristige Maßnahmen der Regulatoren}

\chapter{Fazit und Handlungsempfehlungen}
\nocite{gs1}

\newpage
\printbibliography[
  heading=bibintoc,
  title={Literaturverzeichnis}
  ]

\end{document}
http://web.mit.edu/rsi/www/pdfs/new-latex.pdf
http://web.mit.edu/rsi/www/pdfs/bibtex-format.pdf
http://www.ir.rochelleterman.com/sites/default/files/Helleiner%202011.pdf
Quelle 1: https://www.federalreserve.gov/econresdata/releases/mortoutstand/mortoutstand20090331.htm
Quelle 2: https://web.archive.org/web/20090419051155/http://www.sifma.org/uploadedFiles/Research/Statistics/SIFMA_USBondMarketOutstanding.pdf
Quelle 3: https://fraser.stlouisfed.org/timeline/financial-crisis
Quelle 4 (Doss, 2007): https://www.wsj.com/articles/SB116775515272364964
Quelle 5 (Creswell \& Bajaj, 2007): https://www.nytimes.com/2007/04/03/business/03lend.html
https://academic.oup.com/cje/article/33/4/563/1730705
GS: https://www.goldmansachs.com/insights/pages/learning-from-a-century-us-recessions/report.pdf
Buffett: https://www.youtube.com/watch?v=MQcPC31KRqA
Dimon: https://www.youtube.com/watch?v=QE3QwTA5ujE
Dimon text: https://www.jpmorganchase.com/corporate/investor-relations/document/annualreport-2018.pdf
EU: https://europa.eu/rapid/press-release_MEMO-08-123_en.htm?locale=en
ä \"a
ö \"o
ü \"u
ß
(28.09.19)
Ap. 1: mit Macro Bond selbst erstellt
Ap. 2: https://commons.wikimedia.org/wiki/File:P060708_22.03-02-retouched.jpg
Ap. 3: https://www.advisorperspectives.com/images/content_image/data/6c/6c77ccb14b1810a345ed768e3f8d5272.png