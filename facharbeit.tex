\documentclass[a4paper,draft]{report}
\usepackage[ngerman]{babel}
\begin{document}
\title{Wurden die richtigen Lehren aus der
Finanzkrise 2007-2009 gezogen?}
\author{Gabriel Tomitsuka}
\date{Oktober 2019}
\maketitle
\tableofcontents
\section{Einleitung}
Am 2. April 2007 meldete der Real Estate Investment Trust (REIT) New Century die Insolvenz an.
Wenige Monate davor hatte er eine
Marktkapitalisation von \$1.75 Milliarden.
<story zu anfang>
\subsection{Themawahl}
Ich habe mich oft gefragt, wie ein Markt, der
von kleinen, amerikanischen Regionalbanken mit
staatlicher Unterstützung
dominiert wird, zum Fall von als "Too Big To
Fail" verstandene Investmentbanken wie Lehman
Brothers und Merill Lynch
führen konnte. Besonders interessant fand ich,
dass auch Banken in anderen Kontinenten (wie
bspw. die Dresdner Bank) dadurch
aufgekauft werden mussten. Ich wollte genauer
ermitteln, wie stark diese Vernetzung damals
war und heute ist, und ob
die kurzfristigen Maßnahmen durch
Zentralbanker, Banken/Fonds und Regulatoren
womöglich eine noch schlimmere Krise
vermieden haben.
Allerdings werde ich mich nicht besonders
lange mit einer historischen Betrachtung
beschäftigen, ob die kurzfristigen
Maßnahmen wie geplant wirkten, sondern viel
mehr mit der Fragestellung, ob die mittel- und
langfristigen Maßnahmen die Welt
auf die nächste Rezession richtig vorbereitet
haben --- alle Händler und Investmentbanker, mit
denen ich mich im Rahmen meiner Facharbeit ausgetauscht habe, waren
der Ansicht, dass wir uns spät im Marktzyklus
befinden, und die Frühindikatoren zeigen
ähnliche Signale.
Daher ist es besonders wichtig zu
überprüfen, inwiefern das gesamtwirtschaftliche System auf die nächste Rezession vorbereitet ist.
\subsection{Vorgang}
Die Ursachen für die Finanzkrise 2008 können 
in zwei Gruppen unterteilt werden: realwirtschaftliche
Ursachen der Immobilienkrise und die darausfolgenden
finanzwirtschaftliche Ursachen der globalen Finanzkrise.
Ueber die realwirtschaftlichen Ursachen besteht weithin
Einigkeit innerhalb der Wirtschaftswissenschaften, daher
werde ich verkuerzt auf sie eingehen.
Um die finanzwirtschaftliche Ursachen zu erläutern, werde
ich erst auf die theoretische Grundlage des Finanzsystems
nach den Deregulierungen in den 1970er Jahren eingehen,
sowie die der Bankentheorie. Daraufhin werde ich die Instrumente
untersuchen, die dazu beigetragen haben, dass diese theoretische Grundlage
nicht mehr galt --- dazu zaehlen Asset-Backed Securities (ABS),
Collateralized Debt Obligations (CDO) und
Credit Default Swaps (CDS). Daraufhin werde
ich untersuchen, wie genau sie für die
Krise mitverantwortlich sind. Ferner werde ich
die Frage erörtern, ob es andere Instrumente
oder Anlageklassen gibt, die eine
besondere Wirkung auf die Krise hatten, die in
der Öffentlichkeit nicht angekommen ist.
Daraufhin werde ich die kurzfristige Reaktion
der Regulatoren und Zentralbanker ab 2008
untersuchen. Genauer werde ich auf
Sofortmaßnahmen wie die kontroverse
Bankenrettung, (bis heute anhaltende)
Zinssenkungen, <etc> eingehen.
Schließlich werde ich auf die mittel- und
langfristigen Maßnahmen, inkl. Basel
III, CRD und <etc> eingehen. Außerdem
werde ich mit Bezug auf Gesprächen mit
Händlern, Risikoverwaltern und
Investmentbankern erörtern, ob diese
Maßnahmen die europäische
Wirtschaft ausreichend auf die
Herausforderungen der Zukunft vorbereiten.
\end{document}
http://web.mit.edu/rsi/www/pdfs/new-latex.pdf
http://web.mit.edu/rsi/www/pdfs/bibtex-format.pdf
http://www.ir.rochelleterman.com/sites/default/files/Helleiner%202011.pdf
Quelle 1: https://www.federalreserve.gov/econresdata/releases/mortoutstand/mortoutstand20090331.htm
Quelle 2: https://web.archive.org/web/20090419051155/http://www.sifma.org/uploadedFiles/Research/Statistics/SIFMA_USBondMarketOutstanding.pdf
https://academic.oup.com/cje/article/33/4/563/1730705
https://www.goldmansachs.com/insights/pages/learning-from-a-century-us-recessions/report.pdf

ä
ö
ü
ß
